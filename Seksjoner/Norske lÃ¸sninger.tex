\section{Norske løsninger}

Det norske markedet for applikasjoner som hjelper deg på reise er ikke like modent som det internasjonale markedet. Ingen enerådende løsninger finnes, men vi har sett på 14 applikasjoner i det norske markedet. Disse applikasjonene finnes enten i Google Play, Apple App Store eller som enkeltstående webapplikasjoner.

Crowdsourcing er blitt populært for å samle inn informasjon fra brukere. Internasjonalt har flere applikasjoner dratt god nytte av dette. Det viser seg at i Norge er ikke brukermassen stor nok til at disse får noen særlig nytteverdi. Ingen norske løsninger benytter seg av denne teknologien. NRK P4s applikasjon viser informasjon fra innringte trafikkmeldinger, og det er mulig å registrere en trafikkulykke eller kø. Dette er det nærmeste man har crowdsourcing i det norske markedet.

Tilgangen på trafikkdata i Norge begrenser funksjonaliteten til applikasjonene. De fleste appene benytter seg av de samme dataene, og viser denne på forskjellige måter. Disse dataene er:

\begin{itemize}
\item Trafikkmeldinger fra Statens Vegvesen, Trafikksentralen

\item Bilder fra vegkameraene til Statens Vegvesen

\item Data om trafikktetthet på teststrekninger fra Statens Vegvesen

\end{itemize}

Alle applikasjonene er svært like, og lite skiller de fra hverandre. Dette bunner ut fra den samme begrensede tilgangen på data. De viser trafikkmeldingene fra SVV på et kart, med noe tilhørende metadata. Vår mening er at det ikke finnes noen komplett reiseassistent-applikasjon i Norge.

\subsection{Undersøkte applikasjoner}

Applikasjonene finnes på Google Play eller Apple App Store

\begin{itemize}
\item Trafikk - av NAF og VG

\item iTrafikken - Ciber Norge AS

\item Trafikk Melding -

\item Trafikkinfo

\item Trafikkflyt - P4

\item Traffikkflyt Bergen

\item Trafikk - Frank Burmo

\item Trafikk Widget - Unixcrap Apps

\item VegAppen - Sveinung Kval Bakken

\item Statoil

\item Trafikkmelding

\item Dit.no

\item Google Maps

\item Apple Maps

\end{itemize}
