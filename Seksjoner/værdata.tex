\section{Værdata}
\paragraph{Værdata internasjonalt er gjerne samlet inn av private selskaper og solgt til de som er interessert. I Norge kan man derimot finne gratis tilgjengelig data fra flere kilder, som yr.no(en av grunnene til at yr.no også er mye brukt i utlandet). Vi har sett nærmere på 3 forskjellige kilder, yr.no, storm.no og Meterologisk Institutt.}

\subsection{yr.no}
\paragraph{yr.no har tilgjengeliggjort all sin data via et åpent API, noe som vil i at hvem som helst kan hente data herfra. Det finnes visse restriksjoner på hvor ofte man kan hente ut data(hvert 10 min sånn ca), men dette begrenser ikke vår eventuelle bruk av tjenesten. Tjenesten fungerer ved at man sender inn en lokasjon og får tilbake en værmelding for denne lokasjonen. Kan hente informasjon om værforhold ved fjelloverganger.}

\subsection{Meterologisk Institutt}
\paragraph{Meterologisk institutt tilbyr mye de samme dataene som yr.no(mener at yr.no henter sie data fra nettopp Meterologisk Institutt), og har også et eget API man kan kommunisere med. Tilbyr også mye mer detaljerte data. Informasjonen er hentet fra værstasjoner rundt om i landet(tror det er rundt 2000 av dem).}

\subsection{storm.no}
\paragraph{Storm.no tilbyr dessverre ikke et åpent API for å hente ut data(mener at Storm tjener mye av sine penger på å nettopp selge denne informasjonen). De tilbyr derimot en tjeneste som ligner litt på det vi har tenkt på, nemlig trafikkvær. Her kan man planlegge en rute på et kart og få en værmelding langs denne ruten. På kartet er det også lagt inn veimeldinger(ser ut som de henter det direkte fra Vegvesenet). Vi kan dessverre ikke bruke denne tjenesten i en eventuell app, men det viser at det er mulig, og at det finnes interesse for slik informasjon.}